\chapter{Background}
\todo{Need to rewrite this section: give background for faceted search and faceted web search. Move some related work to their corresponding Chapters}
\todo{Compare two search paradigms}
\todo{Information/Knowledge representation: taxonomy, faceted taxonomy, ontology}
\todo{Focalized search vs. exploratory search}
\todo{Set retrieval and ranked retrieval}
\todo{Faceted navigation and faceted search}
\todo{Other related techniques: query suggestion, search result clustering/organization, search result diversification}
%In this chapter, we review related work for faceted web search.
\label{ch:bg}
\section{Faceted Search}
Previous work on faceted search has studied automatic facet generation ~\cite{dakka2008automatic,li2010facetedpedia,stoica2007automating,oren2006extending,kohlschutter2006using,latha2010afgf} and facet recommendation for a query~\cite{dash2008dynamic,koren2008personalized}. Most of the work is based on existing facet metadata or taxonomies, and extending faceted search to the general web is still an unsolved problem. The challenges stem from the large and heterogeneous nature of the web~\cite{teevan2008challenges}: because the web is very large, it is
difficult to assign quality facets to every document in the collection and to retrieve the full set of search results and their associated facets at query time; and because the web is heterogeneous, it is difficult to apply the same facets to every search result or every query.
%Different from previous work which generates facets for a entire corpus~\cite{stoica2007automating,dakka2008automatic}, some recent work~\cite{dou2011finding,kong2013extracting} extracts facets for only a query.

Most evaluations for facet generation/recommendation are either based on comparison between system generated and human created facets~\cite{dakka2008automatic,dou2011finding} or user studies~\cite{dash2008dynamic,li2010facetedpedia,stoica2007automating}. However, the former may not exactly reflect the utility in assisting users' search tasks, and the latter is expensive to extend for evaluating new systems. In a similar spirit to ours, some work~\cite{schuth2011evaluation,zhang2010interactive,koren2008personalized} also evaluates facets by their utility in re-ranking documents for users. The differences are their evaluation methods do not capture the time cost for users as explicitly as we do, and their experiments are based on corpora with human created facet metadata. Other evaluations~\cite{burke1996knowledge,english2002hierarchical,hearst2006design,hearst2008uis,kules2009exploratory} for faceted search are mostly done from a user interface perspective, which is beyond the scope of 
this proposal.

\section{Query Subtopic/Aspect Mining}
To address multi-faceted queries, much previous work studied mining query subtopics (or aspects). 
A query subtopic is often defined as a distinct information need relevant to the original query.
It can be represented as a set of terms that together describe the distinct information need~\cite{wang2009mining,wu2011identifying, dang2011inferring} or as a single keyword that succinctly describes the topic~\cite{song2011overview}. 
Different resources have been used for mining query subtopics, including query logs~\cite{wang2007learn,hu2012mining,xue2011topic,wang2009mining,wu2011identifying,yin2010building}, document corpus~\cite{allan2002using} and anchor texts~\cite{dang2011inferring}.
% Much work~\cite{Wang:2007:LWS:1277741.1277759, Hu:2012:MQS:2348283.2348327} uses related queries from search logs as candidates, and clustered them into query subtopics. Wang and Zhai~\cite{Wang:2007:LWS:1277741.1277759}, for example, used snippets of a query's clicked web documents to enrich the query representation, and then cluster related past queries into query subtopics.
%Due to data sparsity for instance-level query subtopics, Some work~\cite{Wang:2009:MBL:1557019.1557114,Xue:2011:TMN:2063576.2063877,Wu:2011:IAW:2016945.2016963,Yin:2010:BTW:1772690.1772792} mined generic query subtopics, which are query subtopics for a generic class of queries.
%For example, Yin et al.~\cite{Yin:2010:BTW:1772690.1772792} built taxonomies of query subtopics for categories of name entity queries using search logs.
%Wu et al.~\cite{Wu:2011:IAW:2016945.2016963} also worked on identifying query aspects for named entities queries. They propagated reformulation phrases for a classes of named entities queries.
%Other than query logs, query subtopics can also be mined from documents. For example, Dang et al.~\cite{Dang:2011:IQA:2063576.2063904} worked on clustering related anchor texts in ClueWeb09 corpus into query subtopics. Allan et al.~\cite{Allan:2002:UPP:564376.564430}, from a text corpus, extracted commonly occurring parts of speech pattern near a single-word query to find different potential specifications of the query.

Query subtopics and facets for a query are different in that the terms in a query subtopic are not restricted to be coordinate terms, or have peer relationships. Facets for a query, however, organize terms by grouping ``sibling'' terms together. For example, \{\textit{news}, \textit{cnn}, \textit{latest news}, \textit{mars curiosity news}\} is a valid query subtopic for the query \textit{mars landing}, which describes the search intent of Mars landing news, but it is not a valid facet. Instead, a valid facet that describes Mars landing news could be \{\textit{cnn}, \textit{abc}, \textit{fox}\}, which includes different news channels.
%In a recent work~\cite{Dou:2011:FDQ:2063576.2063767}, Dou et al. developed a system to extract facets from web search results and showed the potential of doing so. However, the unsupervised method they proposed is far from optimal, and it does not improve by having human labels available. Also, to the best of our knowledge, their evaluation can be problematic in some cases, which will be discussed in Section~\ref{sec:evalmetricsall}.

\section{Semantic Class Extraction}
Semantic class extraction is to automatically mine semantic classes represented as their class instances from certain data corpora. For example, it may extract \textit{USA}, \textit{UK}, \textit{China} as class instances of semantic class \textit{country}. Due to the similar semantic relationships between terms inside a facet and a semantic class, semantic class extraction can be used for facet generation. Existing approaches can be roughly divided into two categories: distributional similarity and pattern-based~\cite{shi2010corpus}. The distributional similarity approach is based on the distributional hypothesis~\cite{Harris}, that terms occurring in analogous contexts tend to be similar. Different types of contexts have been studied for this problem, including syntactic context~\cite{pantel2002discovering} and lexical context~\cite{pantel2004towards,agirre2009study,pantel2009web}.
The pattern-based approach applied textual patterns~\cite{hearst1992automatic,pasca2004acquisition}, HTML patterns~\cite{shinzato2007simple} or both~\cite{zhang2009employing,shi2010corpus} to extract instances of a semantic class from some corpus.
The raw semantic class extracted can be noisy. To address this problem, \citet{zhang2009employing} used topic modeling to refine the extracted semantic classes.
Their assumption is that, like documents in the conventional setting, raw semantic classes are generated by a mixture of hidden semantic classes.
In this work, we apply pattern-based semantic class extraction on the top-ranked Web documents to extract candidates for query facet generation.

\section{Search Results Diversification}
Search result diversification has been studied as a method of tackling ambiguous or multi-faceted queries while a ranked list of documents remains the primary output feature of Web search engine today~\cite{agrawal2009diversifying,clarke2008novelty,santos2010exploiting,sakai2011evaluating,dang2013term}.
The purpose is to diversify the ranked list to account for different search intents or query subtopics.
A weakness of search result diversification is that the query subtopics are hidden from the user, leaving him or her to guess at how the results are organized.
FWS addresses this problem by explicitly presenting different facets of a query using groups of coordinate terms for users to select.

\section{Search Result Clustering/Organization}
Search results clustering is a technique that tries to organize search results by grouping them into, usually labeled, clusters by query subtopics~\cite{carpineto2009survey}.
It offers a complementary view to the flat ranked list of search results.
Most previous work exploited different textual features extracted from the input texts and applied different clustering algorithms with them.

Instead of organizing search results in groups, there is also some work~\cite{lawrie2001finding,lawrie2003generating, nevill1999lexically} that summarizes search results or a collection of documents in a topic hierarchy. For example, previous studies~\cite{lawrie2001finding,lawrie2003generating} used a probabilistic model for creating topical hierarchies, in which a graph is constructed based on conditional probabilities of words, and the topic words are found by approximately maximizing the predictive power and coverage of the vocabulary.

FWS is different from these work in that it provides facets of a query, instead of directly organizing the search results. The facet interface allows users to filter/re-rank search results from multiple aspects, instead of a single, taxonomic order.

\section{User Feedback}
There is a long history of using user explicit feedback to improve retrieval performance. In relevance feedback~\cite{rocchio71relevance,salton90improvingretrieval}, documents are presented to users for judgment, after which terms are extracted from the judged relevant document, and added into the retrieval model. In the case where true relevance judgments are unavailable, top documents are assumed to be relevant, which is called pseudo relevance feedback~\cite{buckley1995automatic,abdul2004umass}. Because a document is a large text unit which can be difficult for users to judge and for the system to incorporate relevance information, previous work also studied user feedback on passages~\cite{allan1995relevance,xu1996query} and terms~\cite{koenemann1996case,tan2007term}.

For faceted search, previous work~\cite{zhang2010interactive} studied user feedback on facets, using both boolean filtering and soft ranking models. However, the study is based on corpora with human created facet metadata, which is difficult to obtain for the general web. One other difference between our work and most other user feedback work is, facet feedback in our work is used to improve ranking with respect to the query subtopic specified by the feedback terms, instead of the query topic represented by the original query. This presents the scenario in FWS, where users start with a less-specified query, and then use facets to help clarify and search for subtopic information.

\section{Summary}
Faceted web search (FWS) we proposed in this work is different from all the past work. It extends conventional facet search from a commonly fixed-domain setting to an open-domain web setting. It is different from search result diversification in that instead of hiding those query subtopics from users, it explicitly presents different facets of a query. It is also different from search results clustering or organization in that instead of directly organizing the search results, the facet interface in FWS allows users to filter/re-rank search results from multiple aspects.

We studies three main issues of FWS that have not been explored in previous work, including facet generation, facet feedback and evaluation for FWS. Facet generation for FWS is different from query subtopic mining due to the different nature of query subtopics and facets. It is also different from semantic class extraction in that it targets for a general web query instead of a semantic class. Facet feedback for FWS is different from other user feedback due to their different purposes. Facet feedback targets at improving ranking with respect to the ``query subtopic'' specified by the feedback terms, instead of the query topic represented by the original query. Our evaluation for FWS are also different from previous ones for faceted search. We consider many different aspects (e.g., cost and gain), and the evaluation is based on simulations instead of user studies, which makes it relative cheap to extend for new systems.
